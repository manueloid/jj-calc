\section{STA and Josephson Junction}
In this section we will outline the calculations that allow STA to be applied in these Josephson Junction settings.
%Full Hamiltonian of the system
Starting from \cref{eq:HamiltonianTwoModeFinal} it can be shown that the operators $ \hat{n} $ and $ \hat{ \alpha } $ behave as pseudoangular momentum operators if we set  
\begin{equation}
\label{eq:PseudoAngularMomentumDefinition}
\hat{J}_z = \frac{\hat{c}_{r}^{\dagger}\hat{c}_{r} - \hat{c}_{l}^{\dagger}\hat{c}_{l}}{2} = \hat{n},\hspace{3em} \hat{J}_x =\frac{\hat{c}_{r}^{\dagger}\hat{c}_{l} - \hat{c}_{l}^{\dagger}\hat{c}_{r}}{2} =  \hat{ \alpha }  ,\hspace{3em}  \hat{J}_y = \frac{\hat{c}_{r}^{\dagger}\hat{c}_{l} - \hat{c}_{l}^{\dagger}\hat{c}_{r}}{2i} 
\end{equation}
and we obtain the Bose Hubbard Hamiltonian 
\begin{equation}
\label{eq:HamiltonianJosephsonJunctionSTA}
H_{BH} =  U \hat{J}_z^2 - 2J\hat{J}_x 
\end{equation}
if the value $ U $ and $ J $ are chosen accordingly.
It can also be shown that the operators defined in \cref{eq:PseudoAngularMomentumDefinition} follow in fact the angular momentum relations.\\
%Description of the basis states 
By using the pseudoangular momentum approach, a system of N particles can be described as a single particle with spin $ N/2 $ and the basis set is of the form $ \{\ket{m}\} $ with $ m = -N/2, ..., N/2 $ eigenstates of the $ \hat{J}_{z} $ operator.\\
The Hamiltonian of the system is then defined via \cref{eq:HamiltonianJosephsonJunctionSTA} and the general state $ \ket{\Psi} $ can be written as 
\begin{equation}
\label{eq:GeneralStateAngularMomentum}
\ket{\Psi} = \sum_{m = -N/2}^{N/2} c_{m}\ket{m}.
\end{equation}
% Schr{\"o}dinger equation 
The Schr{\"o}dinger equation is then written as 
\begin{equation}
\label{eq:SchrodingerEquationBoseHubbardAngular}
	i\partial_t \ket{\Psi} = H_{BH}\ket{\Psi} 
\end{equation}
If we want to apply STA to \cref{eq:SchrodingerEquationBoseHubbardAngular}, we need to perform some approximations.
In the following I will try to perform the same approximation they used in \cite{BoseEinsteinCJulia2010} in order to move from the discrete to the continuous variable.
I will follow the calculations I found in \cite{BoseEinsteinCJulia2010} as they give a better idea on what is the Hamiltonian of the system and what are the steps and approximations we need to make in order to obtain an idealised version of the Hamiltonian where we can apply STA.
My plan is to obtain the idealised version of the Hamiltonian they used in \cite{FastGenerationJulia2012}

%First approximation 
The first thing to do is to define a new dimensionless Hamiltonian $ H_{S} = \frac{H_{BH}}{NJ} $ that reads
\begin{equation}
\label{eq:DimensionlessBoseHubbardHamiltonian}
	H_{S} = -\frac{2}{N}\hat{J}_x +\frac{U}{NJ}\hat{J}_z ^{2} = -\frac{2}{N} \hat{J}+ \frac{2\Lambda}{N^2}\hat{J}_z^2 
\end{equation}
where we defined $ \Lambda = NU/(2J) $. 
The corresponding Schr{\"o}dinger equation then becomes $ \frac{i}{NJ}\partial_t\ket{\Psi} = H_{S}\ket{\Psi} $ and if we introduce the dimensionless time $ \tau = t/J $, it simplifies even more, becoming
\begin{equation}
\label{eq:SchrodingerEquationDimensionless}
	\frac{i}{N}\partial_{\tau} \ket{\Psi} =
	\left( 
		-\frac{2}{N} \hat{J}_{x}+ \frac{2\Lambda}{N^2}\hat{J}_z^2 
	\right)\ket{\Psi}
\end{equation}

%Applying angular momentum operators  
We now want to find a differential equation for the coefficients $ c_{m} $ of \cref{eq:GeneralStateAngularMomentum}.
In order to do that, we are going to project \cref{eq:SchrodingerEquationDimensionless} onto $ \bra{m} $.
Moreover, we are going to use the fact that $ \hat{J}_x = \frac{1}{2}\left(\hat{J}_{+} + \hat{J}_{-}\right) $.
If we are to project onto $ \bra{m} $ we should remember how the ladder operators $ \hat{J}_{\pm} $ act.
In particular, we can see we are only interested in those states $ \ket{k} $ such that $ \hat{J}_{\pm}\ket{k} = \beta_{k} \ket{m} $ with $ \beta_{k} $ some coefficient depending on the quantum number $ k $.
We are interested in such states as they are the ones which the projection on $ \bra{m} $ is non zero and we can see that for a fixed $ m $   only the states $ \ket{m \pm 1} $ meet the requirements.
Take for example $ \ket{m +1} $ then
\begin{equation}
\label{eq:LadderOperatorsProjected}
\braket{m|\hat{J}_{-} |m + 1 } = \sqrt{
	\left(\frac{N}{2} + m +1\right)
	\left(\frac{N}{2} -m\right)
} 
	\braket{m|m}
	= 
	\beta_{m}
\end{equation}
where we set $ \beta_{m} =\sqrt {    \left(\frac{N}{2} + m +1\right)  \left(\frac{N}{2} -m\right) }$.\\
% Discrete formulation for the Schr{\"o}dinger equation
By putting everything back together, we obtain
\begin{align}
	&	\braket{m|\frac{i}{N}\partial_t|\Psi} = \braket{m|\tilde{H}_{S}|\Psi}\label{eq:SchrodingerEquationCoefficientsDiscrete1}\\
	& 	\frac{i}{N}\frac{d}{dt}c_{m}(t) =  -\frac{2}{N}\left( b_{m}c_{m+1}(t) + b_{m-1}c_{m-1}(t) \right) + \frac{2\Lambda}{N^2}m^2c_{m}(t)\label{eq:SchrodingerEquationCoefficientsDiscrete2}
\end{align}
where in this case we set $ b_{m} = \beta_m/N $. 
The result in \cref{eq:SchrodingerEquationCoefficientsDiscrete2} gives us a Schr{\"o}dinger equation for the coefficients $ c_{m} $ which is discrete. We now need to move from a discrete formualtion to a continuous one and we are going to do that by performing a change of variable.
If we look at the definition of $ b_{m} $, we see that we can collect the $ N/2 $ term as shown in the following
\begin{equation}
\label{eq:CoefficientCollectingN}
 b_m =\frac{1}{N}\sqrt{
 		\left(\frac{N}{2} + m +1\right)  \left(\frac{N}{2} -m\right) 
	}  =
 	\frac{1}{N}\sqrt{
		\frac{N^2}{4}    
		\left(1 + \frac{m}{N/2} +\frac{1}{N/2}\right)
		\left(1 -\frac{m}{N/2}\right) 
	}
\end{equation}
and if we define the continuous variable $ z = \frac{m}{N/2} $  and $ h = \frac{1}{N/2} $, we obtain
\begin{equation}
\label{eq:CoefficientContinuous}
\frac{1}{2}
	\sqrt{ 
		\left(1 + z +h\right)
		\left(1 -z\right) 
	}
	:=  b_{h}(z)
\end{equation}
where we can see that $ b_{h}(z-h) = \sqrt{ \left(1 + z \right)\left(1 -z -h\right) } $ which can be mapped back to $ b_{m-1} $.
Additionaly, if we define $ \sqrt{N/2}  c_{m}= \psi(z) $ we see that $ \psi(z \pm h) $ can be mapped to $ c_{m\pm 1} $.
Finally, by recalling that for a function $f(x)$ we have $ f(x \pm \epsilon) = e^{\pm \epsilon\partial_{x}}f(x) $ we can rewrite \cref{eq:SchrodingerEquationCoefficientsDiscrete2} as 
\begin{equation}
\label{eq:SchrodingerEquationCoefficientsContinuous}
	\cancel{\frac{1}{2}}ih\partial_t\psi(z) = 
	-\cancel{\frac{1}{2}}[ e^{-i\hat{p}} b_{h}(z) + b_{h}(z)e^{i\hat{p}} ]\psi(z) +
	\cancel{\frac{1}{2}}\Lambda z^2 \psi(z)
\end{equation}
where $ \hat{p} = -ih\partial_z $.
If we want to mimic the calculations made in \cite{FastGenerationJulia2012} we need to perform a Taylor expansion of both the $ e^{\pm i\hat{p}} $ part and the $ b_{h}(z) $ function up to the second order in $ h $ such as
\begin{align}
&	e^{-i\hat{p}}  \simeq 1 \pm h \partial_{z} - \frac{1}{2}h^2\partial^2_{z} \label{eq:TaylorExpansionExp }\\
&	 b_{h}(z) \simeq  1 + h \partial_{h}b_{h}(z)|_{h = 0} + \frac{1}{2}h^2\partial_h^2b_{h}(z)|_{h = 0} \label{eq:TaylorExpansionB }.
\end{align}
By carrying out the calculations,we obtain the following Schr{\"o}dinger equation
\begin{equation}
\label{eq:TaylorExpansionHamiltonian}
	ih\partial_t\psi(z) = 
	-h^2\partial_z\left(b_{0}(z)\partial_z\psi(z) \right)+
	\left[ 	\Lambda z^2 - 2b_{0}(z) \right] \psi(z)
\end{equation}
where $ b_{0}(z) = \sqrt{1-z^2}  $.
We can retrieve equation (7) in \cite{FastGenerationJulia2012} by setting a new $ \tilde{h} \equiv h/2 = 1/N $.
In the following we will stick to definition of $ h = 1/N/2 $ instead of using the other definition.\\
The last approximation we need to perform in order to obtain an oscillator-like Schr{\"o}dinger equation for this system is given by neglecting the $ z $ dependence of the effective mass term and expanding the $ \sqrt{1 - z^2}  $ term into $ 1 - z^2/2 $ in the external potential term. We can finally write down the Schr{\"o}dinger equation 
\begin{equation} 
\label{eq:SchrodingerEquationHarmonicLike}
	ih\partial_t\psi(z) = H_{ho}\psi(z)
\end{equation}
where the Hamiltonian of the system is given by 
\begin{equation}
\label{eq:HamiltonianJosephsonJunctionHarmonic}
	H_{ho} = -h^2\partial_z^2 + (1 + \Lambda)z^2  = 
	 -h^2\partial_z^2 + \frac{1}{4}\omega^2 z^2  
\end{equation}
if we set $ \omega^2 \equiv 4(1+\Lambda) $.
