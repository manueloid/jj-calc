%eSTA corrections definition
\begin{equation}
	\label{eq:eSTAcorrections}
	-\frac
	{\left( \sum_{n=1}^{\mathcal{N}} |G_{n}|^2 \right)	\left[ \sum_{n=1}^{\mathcal{N}}\text{Re}(G_{n}^{*}\vec{K}_{n})\right]  }
	{\left|\sum_{n=1}^{\mathcal{N}}\text{Re}(G_{n}^{*}\vec{K}_{n})\right|^2 }
\end{equation}
where $ \mathcal{N} $ is the number of STA wavefunctions we take into account.
All the following calculations can be found in \cite{1912.06057v1}
%Calculation of Gns
\subsection{Calculation of $G_{n}$}
In order to calculate the eSTA corrections, we need to evalute the difference between the two Hamiltonians
\begin{equation}
	\label{eq:DeltaH}
	\Delta H = H_{N} - H_{ho} =
	- e^{-i\hat{p}} b_{h}(z) - b_{h}(z)e^{i\hat{p}} + h^2 \partial_z^2 - z^2
\end{equation}
where we can see that the control parameter $ \omega $ is cancelled out.\\
The correction numbers $ G_{n} $ are then given by
\begin{equation}
	\label{eq:Gns}
	G_{n} = \int_{0}^{t_{f}} dt \braket{\chi_{n}(z,t)| \Delta H|\chi_{0}(z,t)}.
\end{equation}
Since we are only interested in the effects of $ \Delta H  $ when applied to the ground state of the STA wavefunctions $ \ket{\chi_{0}} $, we can expand the exponential parts in \cref{eq:DeltaH} and obtain
\begin{equation}
	\label{eq:DeltaHOnGroundState}
	\Delta H \chi_{0}(z,t) = - b_{h}(z-h)\chi_{0}(z-h,t) - b_{h}(z)\chi_{0}(z+h,t) + h^2\partial^2_{z}\chi_{0}(z,t) - z^2\chi_{0}(z,t).
\end{equation}
We can exploit some symmetries of the system in order to avoid calculating some integrals.
In particular due to the parity of the STA wavefunctions \cref{eq:STAWavefunctionBoseHubbard} we can show that both $ G_{2n+1} $ and $ \vec{K}_{2n +1} $ are identically 0 for $ n \in \mathbb{N} $ thus saving us a lot of time and effort.
%Calculation of Kns
\subsection{Calculation of $ \vec{ K }_{n} $ }
%Definition of Kns
The quantity $ \vec{ K }_{n} $ can be evaluated using the following formula
\begin{equation}
	\label{eq:Kns}
	\vec{K}_{n} =  \int_{0}^{t_{f}} dt \braket{\chi_{n}(z,t)| \nabla H_{N}(\vec{ \lambda }_{0}, t)|\chi_{0}(z,t)}
\end{equation}
where $ \nabla H_{N}(\vec{ \lambda }_{0}, t)$ is the gradient of the Hamiltonian of the system with respect to the control parameters.\\
%Adding polynomial
The idea here is to start with the control parameter $ \Lambda $ that works for the STA protocol and add another polynomial $ P_{\vec{\lambda}}(t) $ that would take some values $ \vec{\lambda}  = (\lambda_{1},..., \lambda_{n})$ in the interval for some $ t \in [t_1, t_{n}] $ with $ \lambda_{1} = \lambda_{n} = 0 $, and $ t_{1}  = t_0 , t_{n} = t_{f} $.
We can consider values $ (\lambda_{1},..., \lambda_n) $ as variables and then define a new $ \tilde{\Lambda}(t) = \Lambda(t) +   P_{\vec{\lambda}}(t) $ where $ \nabla H_{N}(\vec{\lambda}_0,t) $ means $ (\partial_\lambda_1 H_{N},..., \partial_\lambda_{n} H_{N}) $   .\\
Since we want to interpolate just for a limited number of points, it is helpful to use the Lagrange interpolation that would take the form
\begin{equation}
	\label{eq:LagrangeInterpolationPolynomial}
	P_{\vec{ \lambda }}(t) = \sum_{j=1}^{n}\lambda_{j}\prod_{\substack{k=1\\k\neq j }}^{n}\frac{t-t_{k}}{t_{j} - t_{k}}.
\end{equation}
It can be simplified even further by recalling that $ \lambda_{1} = \lambda_{n} = 0 $
\begin{equation}
	\label{eq:LagrangeInterpolationPolynomialSimplified}
	P_{\vec{ \lambda }}(t) = \sum_{j=2}^{n-1}\lambda_{j}\prod_{\substack{k=2\\k\neq j }}^{n-1}\frac{t-t_{k}}{t_{j} - t_{k}}.
\end{equation}
%Taking the gradient and see what happens
Recalling the form of $   H_{N} $ from \cref{eq:SchrodingerEquationCoefficientsContinuous} we can see that the only part dependent from $ \lambda_{i} $ is the $ z^2 $ term.
Now taking the gradient of $ H_{N} $ with respect to the control parameters in this case only amounts to perform the following derivatives
\begin{equation}
	\label{eq:GradientLambda}
	\partial_{\lambda_{i}} H_{N}=\partial_{\lambda_{i}} \tilde{ \Lambda }(t)z = \partial_{\lambda_{i}} \left(\Lambda z^2 + P_{\vec{ \lambda }}(t)z^2    \right) =z^2\prod_{\substack{k=2\\k\neq i }}^{n-1}\frac{t-t_{k}}{t_{i} - t_{k}}
\end{equation}
%Actual calculation of the Kns
We are now in a position to calculate the following quantity
\begin{equation}
	\label{eq:KnOverSpace}
	\braket{\chi_{m}|\partial_{\lambda_{i}} H_{N}|\chi_{0}} = \prod_{\substack{k=2\\k\neq i }}^{n-1}\frac{t-t_{k}}{t_{i} - t_{k}} \braket{\chi_{m}|z^2 |\chi_{0}}
\end{equation}
\subsection{Random facts about the integrals}
An important fact about the wavefunctions \cref{eq:STAWavefunctionBoseHubbard} is that they are normalised only if we integrate over the entire real line, instead of the interval $ [-1,1] $.
It does then make sense to perform all the integrals over $ [-\infty, \infty] $.
By doing so, we can see that $ \braket{\chi_{m}|z^2|\chi_{0}} $ and $ \braket{\chi_{m}|\partial_z^2|\chi_{0}} $ are identically zero for $ m \neq 2 $, hence the only integrals we are interested in are
\begin{align}
	%First integral
	  & \braket{\chi_{2}|z^2|\chi_{0}} =
	\int_{\mathbb{R}}dz \chi_{2}^{*}(z,t) z^2 \chi_{0}(z,t) =
	e^{2i\int_{0}^{t}dt\omega_{0}/b^2 }\frac{\sqrt{2}h b   }{\omega_0}\\
	%Second integral
	  & \braket{\chi_{2}|\partial_z^2|\chi_{0}} =
	\int_{\mathbb{R}}dz \chi_{2}^{*}(z,t) \partial_z^2 \chi_{0}(z,t)=
	e^{2i\int_{0}^{t}dt\omega_{0}/b^2 }\frac{\left(\omega_0 - ib\dot{b}\right)^2}{ \sqrt{8} h\omega_0b^2 }
\end{align}

